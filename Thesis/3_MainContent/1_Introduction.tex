\chapter{Einleitung}
PLS CHANGE DONT FORGET OK BYE Bei der Testautomatisierung des BAYOOSOFT Access Managers ist eine Vielzahl an Testfällen ermittelt worden. Diese wurden jedoch nicht einheitlich gespeichert. Unübersichtliche Excel-Tabellen und Datenbanken gefüllt mit Testfällen zusammen mit Word-Dokumenten, die versuchen, die Bedienung zu erklären, sind der aktuelle Stand dieser Dokumentation. Ein möglicher Ansatz, Ordnung in das Chaos zu bringen, wäre es, Testfälle in einer Ontologie zu sammeln und zusätzlich, um diese zugänglicher für den Arbeitsalltag zu machen, semantische Suchen mit der Ontologie zu verknüpfen. Das Ziel der Hausarbeit ist es, eine Ontologie für Testfälle stichprobenartig anhand eines Features des BAYOOSOFT Access Managers zu entwickeln und zu überprüfen, ob eine semantische Suche mit den Metadaten dieser Ontologie möglich ist. Dafür wird zunächst das Thema Ontologie allgemein näher erläutert und dann schrittweise eine Ontologie erstellt, die im zweiten Schritt, nach einem Überblick über das Thema semantische Suchen, auf die hier ermittelten Aspekte untersucht. Anschließend werden die Ergebnisse auf Nutzen und Umsetzbarkeit bewertet.