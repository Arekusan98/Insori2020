\chapter{Was heißt Sinn?}
Um zu verstehen, welche Bedeutung Sinnerfüllung im Arbeitsalltag hat, muss zuerst geklärt werden, wie sich Sinn definieren lässt. Sinn kann als Weg oder Reise betrachtet werden, als Entscheidung, ob auf einer Richtung, in die man im Leben schreitet, Sinnlosigkeit oder Sinn vorliegt. Die Wahrnehmung von Sinn ist subjektiv, „[e]r wird in einer Sache, einer Handlung, einem Ereignis gesehen, von einer Person in einer bestimmten Situation zugeschrieben.“
Sinn ist, was uns wichtig und zweckdienlich erscheint. Die Frage, ob etwas sinnvoll ist wird immer wieder neu aufgeworfen \cite[S.47-48]{Ehresmann.2018}. Sein Zuschreiben ist also dynamisch. Und so wie einzelne Ereignisse und Aktionen als mehr oder weniger sinnvoll durch das Individuum erachtet werden können, so kann auch das Leben durch das Selbst eingeschätzt werden \cite[S.12]{Schnell.2018}.
In den selbst zugeschriebenen Sinn des eigenen Lebens fließen viele Faktoren ein, die den Weg der Bestimmung verändern können. Menschen werden in soziale und kulturelle Umgebungen geboren. Dabei werden sie mit den Sinnvorstellungen ihrer Umgebung konfrontiert, welche Ziele als erstrebenswert und als unerwünscht gelten. Der Mensch „ist ein Wesen, das in das selbstgesponnene Sinngewebe verstrickt ist, wobei [die] Kultur als dieses Gewebe [anzusehen ist]“ \cite[S.47-48]{Ehresmann.2018}. Großer Teil dieses Sinngewebes ist die Arbeit. Wir leben einer Arbeitsgesellschaft, wir identifizieren uns über unsere Jobs und Tätigkeiten und andere Menschen schätzen uns dementsprechend ein. Die hohe Bedeutung der Arbeit für den Stand im Leben resultiert nicht zuletzt aus dem hohen Ansehen der Arbeit in der Gesellschafft. Entsprechend definiert sich das subjektive Sinngefühl zu einem wesentlichen Teil über unsere Berufe \cite[S.47-48]{Ehresmann.2018}.
Sinn ist also etwas subjektives, dynamisches, das dem eigenen Leben unter Einfluss der gesellschaftlichen Normen zugeschrieben wird. Die Arbeit als zentrales Element der Gesellschaft und des Individuums machen einen großen Teil der eigenen Definition aus. Entsprechend ist also von hoher Bedeutung, die Sinnerfüllung in der Arbeit zu gewährleisten. 
