\chapter{Was macht Arbeit sinnvoll?}
Nachdem mit einer Sinndefinition eine Grundlage für weitere Diskussionen geschaffen wurde, wird an dieser Stelle nun Schrittweise erläutert, wie und an welche Art und Weise im Berufsleben Sinn gefunden oder Sinnerfüllung verhindert werden kann. 
\section{Arbeit und Individuum}
Die eindeutigste Eigenschaft, die Arbeit einen Sinn verleiht, ist das Erbringen eines Profits. Es ist hierbei ein Mittel für den Zweck, ein Tauschwert \cite[S.192-193]{Voswinkel.2018}. Oder anders gesagt, Arbeit erscheint uns allein dadurch schon als Sinnvoll, da sie uns Geld einbringt, mit dem wir unser Leben überhaupt erst ermöglichen können. Dies kann allerdings nicht der alleinige Aspekt sein, denn die Existenz ehrenamtlicher Tätigkeiten beweist, dass Menschen nicht nur das Geld sehen, wenn sie Sinnerfüllung in ihrer Arbeit spüren. 
Andere Aspekte die hier Einspielen sind Faktoren wie z.B. die aktive Teilnahme am Berufsleben und Entscheidungsspielräume des Individuums. Engagement im Beruf und Sinnerleben sind eine Wechselwirkung. Desto mehr wir uns in den Beruf einbringen, desto sinnvoller erscheint uns unsere Rolle, was uns wiederum motiviert, mehr Engagement zu zeigen. Das liegt daran, dass das Sinnerleben einer der wichtigsten Motivationsfaktoren ist. Man muss Sinnhaftigkeit in seinem Beruf finden und kann dann feststellen, dass es sich lohnt, Einsatz und Energie zu investieren.\cite[S.205-206]{FluterHoffmann.2018}. Wie zuvor erklärt, ist Sinn etwas subjektives und dynamisches, was von einzelnen Menschen einer Sache zugesprochen werden kann. „[Noch] in die restriktivste, monotonste und belastendste Arbeit [können] subjektive Anteile der Arbeitenden eingehen, ja dass ohne diese subjektiven Anteile die Arbeit nicht leistbar wäre“ \cite[S.79]{SenghaasKnobloch.2008}. Mit diesem Einbringen der eigenen Anteile in eine Tätigkeit versuchen arbeitende Menschen „ihre persönlichen Bedürfnisse und Sinnansprüche mit den Gegebenheiten in Einklang zu bringen.“, also ihrer Arbeit einen Sinn zuzuschreiben\cite[S.66]{Freier.2018}. Das Zuschreiben von Sinn und das Einbringen der subjektiven Elemente in das Berufsleben kann durch verschiedene Arbeitsstrukturierungsmaßnahmen vereinfacht werden. Dazu gehören zum Beispiel teilautonome Arbeitsgruppen, die auf Aufgabenwechsel, Aufgabenvergrößerung und Erweiterung der Entscheidungs- und Kontrollspielräume setzen\cite[S.67]{Freier.2018}. Dadurch kann das Individuum mehr subjektive Aspekte in die Arbeit einbringen.
\section{Kohärenz im Beruf}
Ein weiterer Aspekt der Sinnerfüllung im Beruf ist das Konzept der sogenannten „Kohärenz“. Kohärenz in diesem Sinne beschreibt eine globale Orientierung, die das Ausmaß eines Gefühls des Vertrauens beschreibt. Dieses besteht aus den drei Komponenten des Kohärenzgefühls:
\begin{itemize}
    \item Verstehbarkeit\newline
    Die Verstehbarkeit bezieht sich auf das Ausmaß, in welchem man die inneren und äußeren Eindrücke, die während der Arbeit auf das Individuum wirken, als sinnhaft wahrnimmt.
    \item Handhabbarkeit\newline
    Handhabbarkeit beschreibt das Ausmaß, in welchen das Individuum wahrnimmt, wie viele Ressourcen ihm zur Verfügung stehen, sowohl Fähigkeiten als auch Mittel, um die ihm gegenübergestellten Eindrücke zu bewältigen.
    \item Bedeutsamkeit\newline
    Die Bedeutsamkeit ist die emotionale Komponente des Konstrukts der Kohärenz. Sie bezieht sich auf das Ausmaß, in dem die Anforderungen an das Individuum als solche wahrgenommen werden, „die Anstrengung und Engagement lohnen“
\end{itemize}
\cite[S.201]{FluterHoffmann.2018}\newline
Das Zusammenspiel dieser Komponenten beschreibt die Kohärenz im Beruf, die ein Individuum erlebt und fasst die im vorigen Abschnitt genannten Aspekte der Sinnerfüllung zusammen. 
\section{Qualität der Organisation}
Die Qualität der Arbeit ist ein maßgeblicher Faktor für die Zufriedenheit des Individuums mit seiner Arbeit und dem Engagement, das es der Arbeit entgegenbringen möchte. So fällt es leichter, an einer Aufgabe Spaß zu haben und einen Sinn in ihr zu sehen, wenn sie effizient und ohne scheinbar unnötige Hindernisse durchgeführt werden kann. Weiterhin wichtige Faktoren sind eine faire Arbeitseinteilung und die Nachvollziehbarkeit der Praktiken des Vorgesetzten. Fühlt man sich fair behandelt, steigt die Motivation. Weiterhin sollten Tätigkeiten die Kompetenzen des Individuums beanspruchen und diese fordern, damit das Gefühl erlangt wird, dass die eigenen Fähigkeiten von Bedeutung für den Erfolg einer Arbeit sind. Zusätzlich spielt noch das Verhältnis zu Kolleg*innen eine Rolle. Unterstützung aus dem Team zu erhalten und soziale Kontakte aufbauen zu können, verbessert die erlebte Qualität des Berufs und somit auch den Grad der Sinnerfüllung \cite[S.194]{Voswinkel.2018}\cite[S.205-206]{FluterHoffmann.2018}
\section{Arbeit und Gesellschaft}
Es gibt gesellschaftliche Aspekte, die beim Suchen nach dem Sinn in der Arbeit berücksichtigt werden müssen. So lässt sich in Berufen, in denen konkreter Kontakt zu Kunden besteht, Sinn einfacher aufspüren, da ihre Ausübung direkten Einfluss auf das Wohlbefinden anderer Menschen hat. Zum Beispiel Berufe wie Therapeuten oder Altenpfleger zeigen direkte Resonanz bei Kunden, die dem Individuum das Gefühl geben, eine sinnvolle Tätigkeit auszuüben\cite[S.192-193]{Voswinkel.2018}. Dies lässt sich unter anderem dadurch begründen, dass Menschen ihren Beruf als wesentlichen Teil ihrer Identität betrachten. Man setzt sich fast täglich mit seinem Beruf auseinander und wird auch von anderen ein Stück weit mit der eigenen Arbeit identifiziert\cite[S.193]{Voswinkel.2018}. Dementsprechend liegt natürlich nicht nur ein altruistisches Motiv vor, anderen helfen zu wollen und einen sozialen Nutzen zu erfüllen und somit dem Gemeinwohl beizutragen\cite[S.193]{Voswinkel.2018}, sondern auch der Wunsch, sich durch einen sinnvollen Beruf präsentieren zu wollen. Man sucht das „Gefühl, an der Erstellung gesellschaftlich nützlicher Produkte beteiligt zu sein“\cite[S.193]{Voswinkel.2018}, das Gefühl der Sinnhaftigkeit. Dieses Gefühl kann erreicht werden, wenn man durch die eigene Tätigkeit Ziele erreicht, die einem selbst oder anderem wichtig sind. Sinnhaftigkeit kann also in allen Berufen erlangt werden, solange es ein Gefühl gibt, „einen wichtigen Beitrag zu leisten, jemandem zu helfen, den sie schätzen, zu einem Teil des Ganzen wurden oder Verantwortung [zu übernehmen]“ \cite[S.205-206]{FluterHoffmann.2018}
\section{Sinnerfüllung}
Die in den vorigen Abschnitten beschriebenen Eigenschaften von Arbeit in unterschiedlichen Zusammenhängen beeinflussen das Erleben von Sinn eines Individuums in Bezug auf Arbeit. Diese Aspekte lassen sich in vier Kriterien Aufteilen, die das Konzept der Sinnerfüllung beschreiben. 
\begin{itemize}
    \item Kohärenz\newline
    Nicht zu verwechseln mit der in Abschnitt „Kohärenz im Beruf“ erklärten Bedeutung, bezeichnet Kohärenz vielmehr die Passung und Stimmigkeit innerhalb und zwischen Lebensbereichen in Bezug auf die Arbeit.
    \item Bedeutsamkeit\newline
    Bedeutsamkeit beschreibt die erlebte Wirksamkeit und Resonanz des eigenen Handelns, welche in den vorigen Abschnitten auf gesellschaftliche und individuelle Wirkungen untersucht wurden.
    \item Orientierung\newline
    Unter Orientierung wird das Zusammenspiel des Berufs und der Ausrichtung des eigenen Lebenswegs verstanden. Um Sinnerfüllung zu erreichen muss ein Mensch seinen Lebensweg mit den in seinem Beruf ihm gestellten Forderungen bestreiten können.
    \item Zugehörigkeit\newline
    Zugehörigkeit beschreibt die Selbstwahrnehmung als Teil eines Größeren Ganzen wie Z.B. Familie, Freunde, Berufskollegen, Religion usw.
\end{itemize}
\cite[S.12]{Schnell.2018}
