\chapter{Einleitung}
Die Frage nach dem Sinn der eigenen Tätigkeit ist so alt wie das Arbeiten an sich. Seit vielen Jahrzenten fragen sich die Menschen, warum sie etwas tun und suchen den Nutzen in dem, mit dem Sie sich fast täglich auseinandersetzen: ihrem Beruf. Seit Beginn der Industrialisierung und somit der Entstehung der Fließbandproduktion und dem Beginn der Substitution des Menschen in der Produktion wird diese Frage nach dem Sinn immer wieder erschüttert. Seit einigen Jahren bestimmt nun noch ein neuer Faktor die Arbeitswelt. Die Digitalisierung schreitet scheinbar unaufhaltsam voran, doch die möglichen Auswirkungen auf unser Berufsleben und die Sinnerfüllung sind bestenfalls Spekulation und zu einem großen Teil mit einem pessimistischem Blick auf die Entwicklung behaftet. Es gilt also sowohl klarzustellen, was genau uns das Gefühl einer Sinnerfüllung bei der Arbeit gibt als auch wie Sinn sich überhaupt definieren lässt. Dieses Wissen kann genutzt werden, um genau zu verstehen, wo die Risiken und vor allem auch die Chancen der Digitalisierung liegen, die Sinnerfüllung einer Tätigkeit positiv oder negativ zu beeinflussen.