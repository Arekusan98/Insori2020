\chapter{Einleitung}
Die Frage nach dem Sinn der eigenen Tätigkeit ist so alt wie das Arbeiten an sich. Seit Beginn der Industrialisierung und dem Beginn der Substitution des Menschen in der Produktion durch Maschinen wird diese Frage nach dem Sinn jedoch immer wieder erschüttert. Seit einigen Jahren bestimmt nun noch ein neuer Faktor die Arbeitswelt. Die Digitalisierung schreitet scheinbar unaufhaltsam voran, doch die möglichen Auswirkungen auf unser Berufsleben und die Sinnerfüllung sind bestenfalls Spekulation und zu einem großen Teil mit einem pessimistischem Blick auf die Entwicklung behaftet. Es gilt also sowohl klarzustellen, was genau uns das Gefühl einer Sinnerfüllung bei der Arbeit gibt, als auch wie Sinn sich überhaupt definieren lässt. Dieses Wissen kann genutzt werden, um genau zu verstehen, wo die Risiken und vor allem auch die Chancen der Digitalisierung liegen, das Sinnerleben im Beruf zu beeinflussen.