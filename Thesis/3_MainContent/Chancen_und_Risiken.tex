\chapter{Chancen und Risiken durch Digitalisierung}
Die Zukunft der Arbeit ist gestaltbar und gestaltungsbedürftig. Es gibt keinen technologischen Determinismus – die Folgen der unter „Industrie 4.0“ subsumierten neuen
Technologien für Arbeitswelt und Arbeitsmarkt entstehen nicht unmittelbar aus Merkmalen dieser Technologien selbst, sondern aus den Anwendungs- und Einsatzmodellen
für diese Technologien, aus Modellen und Szenarien der Gestaltung von Arbeit auf
gesellschaftlicher, organisationaler und individueller, auf den Einzelarbeitsplatz bezogener Ebene.
%1 Botthof, A., & Hartmann, E. A. (Hrsg.) (2015). Zukunft der Arbeit. Berlin.
2 S. Wischmann und E.A. Hartmann
• Hinsichtlich dieser Folgen für Arbeitswelt und Arbeitsmarkt besonders bedeutsame
Aspekte dieser Anwendungs- und Einsatzszenarien beziehen sich auf die zu Grunde
liegenden Organisationsmodelle. Hier können zwei paradigmatische Modelle unterschieden werden. Das eine Modell stellt eine Substitution menschlicher Arbeit durch
Technik in den Vordergrund (Automatisierungsszenario). Die Aufgabengestaltung
orientiert sich an weitgehender Arbeitsteilung, so werden etwa operative und dispositive Aufgaben in der Regel verschiedenen Beschäftigten zugewiesen. Die Qualifikationsstruktur im Unternehmen tendiert zur Polarisierung: Hoch Qualifizierten auf der
einen Seite stehen niedrig Qualifizierte auf der anderen Seite gegenüber; diese Spaltung
vergrößert sich in diesem Szenario tendenziell.
• Ein alternatives Modell betrachtet Technik eher als Mittel zur Unterstützung und Verstärkung menschlicher Fähigkeiten (Werkzeugszenario). Die Aufgabenteilung ist hier
weniger stark ausgeprägt, operative und dispositive Tätigkeiten werden stärker gemischt,
insbesondere in dem Sinne, dass operative Tätigkeiten mit dispositiven Tätigkeitsanteilen angereichert werden. Die Qualifikationsstruktur im Unternehmen ist weniger stark
polarisiert, es besteht eher ein allgemeiner Trend zur Höherqualifizierung. Niedriger
qualifizierte Tätigkeiten werden entweder durch neue Aufgabenverteilung aufgewertet oder durch Automatisierung obsolet. Das „Füllen von Automatisierungslücken“ mit
menschlicher Arbeit findet sich deutlich weniger als im Automatisierungsszenario.
• In engem Zusammenhang mit diesen Organisationsmodellen lassen sich Prinzipien
progressiver Arbeitsgestaltung für Industrie 4.0 auf unterschiedlichen Aggregationsstufen formulieren. Ein übergeordnetes Gestaltungsprinzip ist die lernförderliche Arbeitsorganisation. Teilaspekte der lernförderlichen Arbeitsorganisation betreffen zunächst
die Vollständigkeit von Arbeitstätigkeiten. Arbeitstätigkeiten sind hierarchisch vollständig, wenn sie anspruchsvolle und Routineaufgaben in angemessenen Anteilen verbinden. Sie sind sequentiell vollständig, wenn planerische, organisierende, ausführende
und kontrollierende Tätigkeitsanteile an einem Arbeitsplatz kombiniert sind. Weitere
Aspekte der Lernförderlichkeit sind Autonomie – Handlungs- und Entscheidungsfreiräume – sowie Transparenz: Kenntnis über die Ergebnisse der eigenen Arbeit und
Kenntnis über andere, verbundene Arbeitsprozesse im Unternehmen.
• Lernförderliche Arbeitsorganisation ist zugleich innovationsförderliche Arbeitsorganisation. Dies hat einmal damit zu tun, dass mit komplexen Aufgaben höhere Kompetenzniveaus der Beschäftigten einhergehen. Das erleichtert sowohl die Wahrnehmung
externer Innovationsimpulse (z. B. neue Bearbeitungsverfahren) wie auch die interne
Verarbeitung, Umsetzung und Verbreitung dieser Innovationsimpulse. Zweitens erhöht
die höhere Lernfähigkeit auch die Wahrscheinlichkeit, dass intern (z. B. im Zuge kontinuierlicher Verbesserungsprozesse) Innovationen erdacht und umgesetzt werden
können.
• Die der Industrie 4.0 zu Grunde liegenden Technologien lassen sich beschreiben als
verteilte, (semi-)autonome, intelligente und vernetzte cyber-physikalische Systeme.
Diese technologischen Grundmerkmale finden sich einerseits in für die Industrie typischen Technologiebereichen (z.  B. Produktionssysteme, Robotik, Logistiksysteme), 
1 Zukunft der Arbeit in Industrie 4.0 – Szenarien … 3
aber auch in anderen Wirtschaftszweigen (z. B. autonome Landmaschinen in der Landwirtschaft, Anwendungen der Künstlichen Intelligenz im Dienstleistungsbereich).
• Die technologischen Eigenschaften der cyber-physikalischen Systeme der Industrie
4.0 implizieren besondere Potenziale einer progressiven Arbeitsgestaltung. Dies liegt
zunächst in der Mächtigkeit und Plastizität dieser Technologien begründet, was sie
für die Umsetzung einer großen Palette möglicher Arbeitsgestaltungsmodelle geeignet erscheinen lässt. Ein besonderer Aspekt der Industrie 4.0 ist der Datenreichtum.
Eine Vielzahl dezentraler und autonomer Sensoren erfasst in Echtzeit große Datenmengen, die durch neue Techniken der Datenverarbeitung analysiert und visualisiert
werden können. Solche Analyse- und Visualisierungstools sind wichtige Elemente
einer Gestaltungsstrategie, die Technik als Verstärker und Unterstützer menschlicher
Fähigkeiten begreift.
• Neben diesen Potenzialen bestehen auch Risiken. Industrie 4.0 ist letztlich auch ein
Automatisierungskonzept und damit anfällig für grundlegende Probleme der Automatisierung. Ein solches grundlegendes Problem zeigt sich in den sogenannten Automatisierungsdilemmata oder -paradoxien: Durch Automatisierung verschieben sich
menschliche Tätigkeiten vom aktiven Steuern der Systeme zur Überwachung automatischer Regelung und zum Einspringen in Situationen, die der Automat nicht beherrscht.
Solche Situationen sind tendenziell eher komplex. Mit einer solchen komplexen Situation sehen sich nun Menschen konfrontiert, die aus zwei Gründen nicht gut darauf
vorbereitet sind, diese Situationen zu meistern. Erstens hat der Mensch diese Situation nicht selbst herbeigeführt und ist deswegen aktuell nicht „im Bilde“; dies ist ein
Problem des Kurzzeitgedächtnisses. Zweitens fehlt durch die Automatisierung die
Übung im aktiven Steuern des Systems und damit degradieren die Fähigkeiten, die zur
Systemsteuerung notwendig sind; dies ist ein Problem des Langzeitgedächtnisses, das
u. a. durch Simulatortraining adressiert werden kann. Für das Problem des Kurzzeitgedächtnisses – das „im Bilde sein“ über den aktuellen Systemzustand – gibt es Gestaltungsmethoden für automatisierte Systeme wie etwas das Ecological Interface Design,
das Möglichkeiten anbietet, den Menschen auf unterschiedlichen Abstraktionsebenen
in der Regelschleife des (semi-)automatischen Systems zu halten
Alle Szenarien, die in den Projekten entwickelt werden, lassen sich mit einem einheitlichen Beschreibungsmodell im Hinblick auf die Implikationen für die Arbeitsgestaltung
darstellen. Dieses Beschreibungsmodell umfasst folgende Kategorien:
4 S. Wischmann und E.A. Hartmann
1. Bedarf: Wie wird sich das technisch-organisationale Szenario (bspw. eine konkrete
Einsatzform von kollaborativen Robotern) auf den Bedarf nach verschiedenen Qualifikationsprofilen (z. B. Facharbeiter, Meister, Ingenieure) auswirken?
2. Hierarchische Vollständigkeit: Wie wird sich das Szenario auswirken hinsichtlich
– Monotonie und komplexen Aufgaben?
– problemlösenden und Optimierungsaufgaben?
– Lernen in der Arbeit?
3. Sequentielle Vollständigkeit: Wie wird sich das Szenario auswirken hinsichtlich
– Planungsaufgaben?
– Kommunikation und Kooperation?
4. Kontrolle und Autonomie: Wie wird sich das Szenario auswirken hinsichtlich
– der Kontrolle des Menschen über seine Arbeitssituation?
– der Selbstbestimmung, den Handlungs- und Entscheidungsspielräumen in der
Arbeitssituation?
5. Querschnittliche und gegenstandsspezifische Aspekte: Wie wird sich das Szenario auswirken hinsichtlich
– der Interdisziplinarität in der Arbeitssituation?
– der Bedeutung von IT-Kenntnissen?
In einem einleitenden Kapitel stellen Steffen Wischmann und Ernst Hartmann diese
Beschreibungsdimensionen vor und präsentieren eine integrierte Betrachtung der Auswirkungen hinsichtlich dieser Dimensionen über alle Praxisprojekte.
In einem ersten Praxisbeispiel stellen Andreas Bächler, Liane Bächler, Sven Autenrieth, Hauke Behrendt, Markus Funk, Georg Krüll, Thomas Hörz, Thomas Heidenreich, Catrin Misselhorn und Albrecht Schmidt Systeme zur Assistenz und Effizienzsteigerung in manuellen Produktionsprozessen der Industrie auf Basis von Projektion
und Tiefendatenerkennung vor. Konkret geht es um ein Assistenzsystem für manuelle
Montageprozesse, das im vom BMWi geförderten Forschungsprojekt motionEAP entwickelt wurde. Neben der technischen Umsetzung werden die pädagogischen, psychologischen und ethischen Aspekte für die Nutzung dieses Assistenzsystems diskutiert.
Ein besonderer Aspekt bezieht sich darauf, wie solche Systeme dazu beitragen können,
leistungsgewandelte und leistungsgeminderte Menschen besser in die Arbeitswelt zu
integrieren.
Ein zweites Szenario bezieht sich auf die industrielle Servicerobotik am Beispiel der
Kleinteilemontage. André Hengstebeck, Kirsten Weisner, Jochen Deuse, Jürgen Rossmann
und Bernd Kuhlenkötter berichten im Kontext des BMWi-geförderten Forschungsprojekts
MANUSERV über die Entwicklung einer webbasierten Planungsumgebung, welche die
Potenziale industrieller Robotersysteme mit den spezifischen Anforderungen manueller
Arbeitssysteme und -prozesse verknüpft. Im konkreten Anwendungsfall wird die Montage
eines Einbauradios mit Touch-Display, das auf Basis einer konkreten Produktspezifikation
des Kunden hergestellt wird, betrachtet. Dabei werden die Potenziale der Nutzung von
Leichtbau-Servicerobotern untersucht.
1 Zukunft der Arbeit in Industrie 4.0 – Szenarien … 5
Im BMBF-geförderten Forschungsprojekt SmARPro (Smart Assistance for Humans in
Production Systems) wird ein System entwickelt, das über einheitliche und standardisierte
Schnittstellen Daten aller umgebenden Systeme erfasst und diese in der SmARPro-Plattform zu kontextsensitiven Informationen aufbereitet. Diese werden dem Mitarbeiter über
Wearables wie beispielsweise Datenbrillen, Smart Watches, Smartphones oder Tablets
angezeigt. Benedikt Mättig, Jana Jost und Thomas Kirks beschreiben Anwendungsfälle im
Bereich der Logistik, wo beispielsweise Kommissionierern und Wareneingangskontrolleuren ihren jeweiligen Rollen und der Situation angepasste Informationen in ihre Datenbrillen eigeblendet werden.
Mit den Auswirkungen von Industrie 4.0 auf die Arbeit in einer Weberei beschäftigen
sich Mario Löhrer, Jacqueline Lemm, Daniel Kerpen, Marco Saggiomo und Yves-Simon
Gloy im Kontext der vom BMBF geförderten Nachwuchsforschungsgruppe SozioTex
an der RWTH Aachen. In dieser Nachwuchsforschergruppe wurde eine interdisziplinäre
Methode zur Entwicklung soziotechnischer Assistenzsysteme in der textilindustriellen
Produktionsarbeit entwickelt. Der konkret beschriebene Anwendungsfall bezieht sich auf
eine Weberei für technische Textilien, wo auf industriellen Webmaschinen beispielsweise
Sicherheitsgurte für die Automobilbranche hergestellt werden. Dort wird ein Assistenzsystem zur Arbeitsunterstützung untersucht, das sich auf ein Tablet als mobiles Endgerät
in Verbindung mit Augmented Reality (AR) stützt.
Roman Senderek präsentiert Ergebnisse des vom BMBF geförderten Verbundprojekts
%ELIAS (Engineering Lernförderlicher Industrieller Arbeitssysteme) anhand der Anwendungsfälle der HELLA KGaA Hueck & Co. und der FEV GmbH, zweier der größten
Unternehmen aus dem deutschen Automotive-Sektor. In dem Projekt wird ein Konzept
entwickelt, das das Lernen im Prozess der Arbeit in bestehende oder zukünftige Arbeitssysteme integriert. Die beiden Beispiele zeigen, dass Maßnahmen zur betrieblichen Weiterbildung eine immer größere Bedeutung gewinnen. Beide Unternehmen reagieren auf
diesen Wandel mit dem verstärkten Einsatz des arbeitsnahen Lernens, in Form unterschiedlicher Lernlösungen.
„Assistenz und Wissensvermittlung am Beispiel von Montage- und Instandhaltungstätigkeiten“ ist das Thema von Carsten Ullrich, Axel Hauser-Ditz, Niklas Kreggenfeld,
Christopher Prinz und Christoph Igel. Im Verbundprojektes APPsist, das vom BMWi
gefördert wird, geht es um die Entwicklung mobiler, kontextsensitiver und intelligentadaptiver Assistenzsysteme, welche die Mitarbeiter beim Wissens- und Kompetenzerwerb
in der Interaktion mit Maschinen auf dem Shopfloor unterstützen. Im konkreten Anwendungsszenario soll der Wechsel eines Werkstoffes in einer teilautomatisierten Montagelinie durch eine angelernte Montagekraft mithilfe von Assistenz durchgeführt werden.
Durch die Assistenzsysteme sollen also an- oder ungelernte Mitarbeiter/Innen dazu befähigt werden, komplexere Prozesse selbstständig und effizient durchzuführen.
Das Projekt InnoCyFer wird vom BMWi gefördert. Susanne Vernim, Christiane Dollinger, Andreas Hees und Gunther Reinhart berichten über die Entwicklung eines Planungsund Steuerungssystems für eine autonome, auf cyber-physischen Systemen basierende
Produktion. Die Produktionsplanung und -steuerung wird hier über einen sogenannten 
6 S. Wischmann und E.A. Hartmann
bionischen Scheduler, der auf einem Ameisenalgorithmus basiert, mit dem physischen
Produktionssystem gekoppelt. Der Produktionsplaner wird über die Planungsvorschläge
oder Steuerungsentscheidungen des bionischen Schedulers informiert und kann bei Bedarf
die gewünschte Möglichkeit auswählen oder bewerten. Dadurch kann er steuernd eingreifen und zu jeder Zeit Produktionsentscheidungen durch eine transparente Aufbereitung
nachvollziehen.
Ebenfalls vom BMWi gefördert wird ReApp – wiederverwendbare Roboterapplikationen für flexible Roboteranlagen; dieses Projekt hat Werkzeuge und Modelle für die Entwicklung wiederverwendbarer Softwarebausteine (Apps) für Roboter zum Gegenstand.
Im Beitrag von Ulrich Reiser, Uwe Müller, Mike Ludwig Mathias Lüdtke und Yingbing
Hue wird die Bestückung von Durchsteckbauelementen auf Leiterplatten als Szenario
betrachtet. Dieser Prozess ist, trotz generell hoher Automatisierungsgrade bei der Bestückung von Leiterplatten, immer noch zum großen Teil Handarbeit. Als Lösung wurde hier
ein automatischer Lötkolben mit Lötdrahtvorschub als Bestandteil einer mobile Einheit
konzipiert, die mit einem Leichtbauroboter, allen nötigen Steuereinheiten und auch
Sicherheitseinrichtungen ausgestattet werden kann.
Alle diese Praxisbeispiele werden nach den am Anfang dieses Kapitals dargestellten
Kriterien bzw. Beschreibungsdimensionen dargestellt. Dadurch wird es möglich, die
unterschiedlichen Szenarien nach einheitlichen Maßstäben zu vergleichen und einen Eindruck über mögliche Implikationen und Auswirkungen von konkreten Implementierungen
von Industrie 4.0 zu gewinnen.
[Zukunft der Arbeit, S.3-6]



Folgt man der Debatte um die Auswirkungen der digitalen Transformation auf Arbeit, wird schnell deutlich,
dass bei aller Automatisierung und Digitalisierung von
Wertschöpfungsketten und Arbeitsstrukturen über
verschiedene Branchen hinweg die Vision der menschleeren Fabrik aus den 1980er Jahren eine solche bleibt
und menschlicher Arbeitskraft weiterhin eine zentrale
Rolle zukommt (Forschungsunion und acatech 2013).
Wie sich industrielle und logistische Arbeit hinsichtlich Gefahren und Chancen entwickeln werden, bleibt
indes ergebnisoffen. Ittermann et al. (2016, S. 13–22;
vgl. auch Hirsch-Kreinsen 2016) formulieren im Rahmen einer Leitbildentwicklung industrieller und logistischer Arbeit im Zuge der Digitalisierung Entwicklungsszenarien digitaler Arbeit3 (. Abb. 15.1). Dieser
sind Anhaltspunkte zu möglichen langfristigen Beschäftigungsperspektiven für beide Wirtschaftssegmente zu entnehmen und zugleich erlauben die Entwicklungsszenarien digitaler Arbeit Aussagen über
mögliche Chancen und Risiken einzelner Beschäftigungsgruppen. Diese Szenarien werden im Folgenden
in höchst zugespitzter Form und als Idealtypen analysiert, um im Anschluss daran eines der Szenarien (Polarisierungsszenario) anhand einer betrieblichen
Kurzfallstudie zu exemplifizieren und an dieser die
Chancen und Risiken für eine als sinn- und bedeutungsvoll erlebte Arbeit aufzuzeigen.
Im Szenario der Substitution von Arbeit (Ittermann
et al. 2016, S. 16–19) konstatieren die Autoren in Bezug
auf die Risiken insbesondere weitreichende Substitutionstendenzen von logistischer und Industriearbeit.
Von dieser und der damit verbundenen Gefahr der
Arbeitsplatzunsicherheit sind insbesondere Beschäftigte betroffen, die in besonderem Maß einfache Tätigkeiten ausführen, die einen hohen Routineanteil und
eine verminderte Handlungskomplexität aufweisen
oder lediglich einen verminderten Schatz an Erfahrungswissen erfordern. Smarte Systeme ersetzen geringqualifizierte und standardisierte „3D-Tätigkeiten“
(dirty, dangerous and demanding) in Produktion und
Logistik, was zum einen als belastungsreduzierende
Chance der Digitalisierung wahrzunehmen ist. Zum
anderen fallen gleichzeitig Aufgabensegmente geringqualifizierter Arbeit weg, indem diese etwa in
Algorithmen übersetzt und dadurch automatisiert
werden. Beispielhaft können hier die Maschinenbedienung für die Produktion und das manuelle Erfassen
und Verwalten von Daten für die Logistik genannt
werden. Einige Arbeitsmarktstudien stützen diese
Substitutionsannahmen (u. a. Frey und Osborne 2013;
Bonin et al. 2015; Dengler und Matthes 2015) und
bescheinigen, dass in allen Wirtschaftsbereichen einfache und zum Teil auch qualifizierte Tätigkeiten in
Bereichen der Planung und Steuerung, der Verwaltung
oder Produktentwicklung durch die Digitalisierung
wegfallen. Eine Chance auf Weiterbeschäftigung haben dem Szenario zufolge wenige hochqualifizierte
Experten (Ingenieure, hochqualifizierte Facharbeiter,
Akademiker), die Wartungsaufgaben der Systeme
übernehmen und daher für Unternehmen unverzichtbar sind. Für sogenannte „Einfacharbeiter“ (Abel et al.
2014) bleiben allenfalls die eben beschriebenen routinisierten Tätigkeiten. Zu erwähnen bleibt, dass für
industrielle und logistische Einfacharbeiten auf Basis
der heterogenen Branchen- und Unternehmensstrukturen durchaus die Chance auf neue Formen digitalisierter Einfacharbeit besteht, beispielsweise durch den
Einsatz von Assistenzsystemen (Niehaus 2017).
Beim Szenario Upgrading von Arbeit (Ittermann et
al. 2016, S. 13–16) lauten die Annahmen, dass mit dem
Einsatz digitaler Technologien (bspw. intelligente Robotersysteme, handlungsunterstützende Assistenzsysteme sowie neue Logistik- und Lagersysteme) v. a.
Chancen für alle Beschäftigtengruppen in Produktion
und Logistik einhergehen. So werden in einschlägigen
Studien zu zukünftigen Beschäftigungszuwächsen
(Boston Consulting Group 2015) die Aufwertung von
Tätigkeiten durch den Technikeinsatz sowie erweiterte
Arbeits- und Handlungsmöglichkeiten vorhergesagt.
Kurzfristige Jobverluste seien durch die Beschäftigungsoptionen, die mit den Technologien einhergehen, zu kompensieren (Wolter et al. 2015). Auf der
Basis wachsenden Wissens über laufende Prozesse
eröffnen sich neue Handlungsspielräume und bestehende Tätigkeiten werden mit neuen Arbeitsinhalten
angereichert. Dem Szenario des Upgrading zufolge
seien „better jobs – jobs that at every level would be
enriched by an informating technology“ (Zuboff 1988,
S. 159) die Konsequenz der Implementierung neuer
Technologien. Diese würden u. a. durch den Einsatz
von Datenbrillen oder Tablets, die auch zur Qualifizierung der Beschäftigten eingesetzt werden können, realisiert. Weitere Chancen sind im „Auf- und Ausbau
von IT-Kompetenzen, Medienkompetenzen und Prozessverantwortung in der Fertigung und Montage,
aber auch in indirekten Bereichen wie der Arbeitsvorbereitung, der Produktionsplanung und der Qualitätssicherung sowie in der Logistik“ (Ittermann et al. 2016,
S. 14) zu sehen. Die Realisierung dieses Kompetenzaufbaus bedarf einer modernen Gestaltung der Lehrpläne in den Berufsschulen und der innerbetrieblichen
Weiterbildungsmöglichkeiten sowie der Umsetzung
(altbewährter) Maßnahmen des „learning-on-the-job“
oder der „job rotation“ (Gebhardt et al. 2015; Spöttl et
al. 2016). In arbeitsorganisatorischer Hinsicht sind insbesondere die dezentrale Anreicherung der Arbeitsinhalte und die Reintegration von vormals getrennten
Arbeitsprozessen als Chancen für die Beschäftigten
anzusehen. Idealtypisch kommt es im Upgrading-Szenario zu einer Reorganisation von Arbeitsformen, die
zukünftig „durch eine lockere Vernetzung unterschiedlich qualifizierter, aber gleichberechtigt agierender Beschäftigter gekennzeichnet ist, die weitgehend
selbstorganisiert und situationsbestimmt“ (Ittermann
und Niehaus et al. 2016, S. 15) in einer digitalen Produktionswelt agieren.
Das dritte Entwicklungsszenario beschreibt die
Polarisierung von Industriearbeit und logistischer Tätigkeiten. Im Gegensatz zum ersten Szenario geht es
von dem Risiko einer partiellen Substitution von Arbeit aus und beinhaltet gleichzeitig die Chance der
Aufwertung von Qualifikation und Kompetenzen,
ähnlich dem Upgrading-Szenario. Zentraler Aspekt
dieses Polarisierungsszenarios ist das Auseinanderklaffen der Lücke zwischen den Beschäftigen, die zum
einen ein hohes Qualifikationsniveau und ein vielseitiges Set an Kompetenzen und Erfahrungen (etwa Ingenieure, Facharbeiter mit Zusatzqualifikation, Supply
Chain Manager etc.) vorweisen, und zum anderen den
Einfacharbeitern, die im Substitutions-Szenario durch
niedriges Qualifikationsniveau, hohen Standardisierungsgrad ihrer Tätigkeiten und Arbeitsteilung sowie
verminderte Entscheidungs- und Handlungsspielräume beschrieben wurden. Zwischen diesen Beschäftigungsgruppen besteht daher die Gefahr, dass die
Ebene der mittleren Qualifikationsebene zunehmend erodiert. Hinsichtlich der Arbeitsorganisation findet
in diesem Szenario eine Ausdifferenzierung statt, die
durch starke Kontrastierung der Beschäftigung der
beiden Gruppen der hochqualifizierten und geringqualifizierten Beschäftigten gekennzeichnet ist.
Durch die Möglichkeit der Automatisierung und Algorithmisierung einfacher und stark strukturierter bzw.
regelorientierter Tätigkeiten können zum Teil komplexe Arbeitsprozesse in mehrere kleine Teiloperationen
gegliedert und zerlegt werden. Dadurch besteht die
Möglichkeit oder negativ betrachtet die Gefahr, dass
eine zunehmend intensivere betriebliche Kontrolle im
Sinne taylorisierter Arbeitsstrukturen stattfindet.
Gleichzeitig ist am Pol der Einfacharbeiten in diesem
Szenario eine vermehrte Einschränkung von Handlungs- und Entscheidungsspielräumen zu befürchten,
die wie oben gezeigt durchaus bedeutungs- und sinnvoll für die individuelle Wahrnehmung von Arbeit
sind. Auf der Kehrseite der Medaille zeigt sich, dass
einige Facharbeitertätigkeiten (bspw. Mechatroniker
oder andere Techniker) eine Aufwertung erfahren und
sich durchaus mit qualifizierteren Tätigkeiten (wie
IT-Fachkräfte oder Ingenieure) vermischen (Spath et
al. 2013 und Boston Consulting Group 2015, S. 9). Das
Polarisierungsszenario birgt zum einen die Gefahr der
Verengung von Handlungs- und Entscheidungsspielräumen bei gestiegenen Kontrollmöglichkeiten und
monotoner Standardisierung der Tätigkeiten. Eine
mögliche Konsequenz dieser Entwicklungen ist, dass
jene un- und angelernten Beschäftigten in bestimmten
logistischen oder industriellen Segmenten schnell ersetzbar sind und die als sinn- und bedeutungsvoll eingeschätzte Arbeitsplatzsicherheit in diesem Szenario
auf der Strecke zu bleiben droht. Zum anderen ist im
Polarisierungsszenario – bedingt durch die Erosion
der mittleren Qualifikationsebene – für einige Beschäftigtengruppen zu beobachten, dass Tätigkeiten
und Arbeitsprozesse aufgewertet werden, indem beispielsweise systemübergreifende Kontroll- und Steuerungsaufgaben im Zuge des Technikeinsatzes auszuführen sind. (Zum Kontext berufliche Stellung und
Sinnerfüllung vgl. 7 Kap. 2)
In der Summe integriert dieses Szenario also widersprüchliche und konträre Perspektiven auf digitale
Arbeit, die zum Teil Chancen, aber auch Gefahren
beinhalten. Den Prognosen folgend sind es zum einen
die Dequalifizierungsprozesse, das hohe Substitutionsrisiko sowie die starke Arbeitsregulierung und Verengung der Handlungs- und Entscheidungsspielräume,
die einer Vielzahl der o. g. Faktoren guter digitaler Arbeit und ihrer sinnvollen Ausgestaltung diametral gegenüberstehen. Zum anderen besteht insbesondere
nicht nur für Facharbeiter und hochqualifiziertes
Personal die Chance der Aufwertung von Tätigkeiten,
sondern auch für einfache Tätigkeiten. Mit der Implementierung neuer Technologien werden Kompetenzen
und Qualifikationen (bspw. IT- und Medienkompetenzen, Datenanalyse und Auswertung, hohes Prozesswissen etc.) (vgl. Ittermann et al. 2016) erforderlich,
die auch zu einer positiven Wahrnehmung des Bedeutungs- und Sinngehalts der Arbeit, wie soziale
Anerkennung, Identifikation mit der Tätigkeit und das
Einbringen neuer Ideen, beitragen.
Gegenwärtig bleibt offen, welcher Entwicklungspfad von Einfacharbeit sich in Produktion und Logistik durchsetzt. Es zeigt sich jedoch, dass das Erscheinungsbild der Einfacharbeit in der Industrie 4.0 differenzierter wird. Offen bleibt dabei ein Zielkonflikt:
Einerseits liegt das Ziel nahe, nach den Perspektiven zu
fragen, wie die Qualität einfacher Arbeit verbessert
und „gute“ Arbeit geschaffen werden kann. Andererseits zeigt sich die arbeitsmarktpolitische Notwendigkeit, dass Einfacharbeit auch als weniger attraktive
Arbeit erhalten bleiben muss, um Beschäftigungsmöglichkeiten für die erhebliche Zahl gering qualifizierter
Erwerbspersonen zu eröffnen. [Fehlzeitenreport S. 182-184]
