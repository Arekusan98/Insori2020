\chapter{Chancen und Risiken der Digitalisierung}
In diesem Kapitel werden nun verschiedene Szenarien der Digitalisierung betrachtet. Nachdem zunächst geklärt wurde, wie Sinnerfüllung in der Arbeit gewährleistet oder gestört werden kann, sollen in den Szenarien die Aspekte untersucht werden, die die Sinnerfüllung positiv oder negativ beeinflussen können.
\section{Substitution von Arbeit}
Das erste Szenario beschreibt den dystopischen Ansatz, dass Digitalisierung zu eine weitreichenden Substitution von Arbeit führt. Viele Berufe und Tätigkeiten, die geringqualifiziert und standardisierbar sind werden von „smarten“ Systeme ersetzt. Besonders betroffen wären hierbei die sogenannten „3D-Berufe“ – „dirty, dangerous and demanding“. Zwar fallen sehr belastende Tätigkeiten weg, aber eben auch viele geringqualifizierte Aufgaben und die damit verbundenen Berufe, zum Beispiel im Bereich der Logistik und Industriearbeit. Zudem ist dieses Szenario in den aktuellen Entwicklungen begründet, da sich Substitutionstendenzen in einzelnen Branchen und Bereichen erkennen lassen. Die Weiterbeschäftigung wäre nur wenigen Hochqualifizierten Berufen vorbehalten, besonders im IT-Bereich, oder Berufen die sich nicht so leicht standardisieren lassen. \cite[S.182-183]{Eisenmann.2018}\\

Dieses Szenario stellt für das Sinnerleben im Beruf für die meisten Betroffenen eine Katastrophe dar. Ganz klar im Vordergrund steht hier natürlich die Angst des Arbeitsplatzverlusts. Nicht nur fällt für die Betroffenen logischerweise das Einkommen aus, wenn ihr Beruf substituiert wird, sondern der Bedarf an ihren Kompetenzen ebenfalls. Wie im vorigen Kapitel erklärt, ist es für das Sinnerleben im Beruf wichtig, dass die Kompetenzen des Individuums in erfüllbarem Maß gefordert und gebraucht werden. Wird die gesamte Branche durch Technik ersetzt, so ist die Botschaft, die beim Individuum ankommt, dass seine Fähigkeiten, sein Wissen nun nicht mehr benötigt werden. Dies greift noch tiefer in das Sinnerleben ein als der Verlust des Einkommens. In Extremfällen wird die gesamte Identität in Frage gestellt. Nicht nur trägt man nicht mehr aktiv seinen Teil zur Gesellschaft bei, sondern scheint zunächst permanent keinen Beitrag jemals wieder leisten zu können. Dies ist nicht zuletzt auch der Orientierung des Lebensweges verschuldet. Hat man zum Beispiel geplant, seinen Beruf lebenslang auszuüben oder sich in einem bestimmten Bereich hochzuarbeiten, den es nach einer vollständigen Digitalisierung vielleicht schlichtweg nicht mehr gibt, ist man zu einer Umschulung in einem so großen Ausmaß gezwungen, die manchmal nicht mehr möglich ist. 
\section{Upgrading of Work}
Upgrading of Work bezeichnet die Utopie der Digitalisierung als Aufwertung der Tätigkeiten. Durch den Einsatz moderner Technologien wie intelligente Roboter- oder Assistenzsysteme können die Arbeits- und Handlungsmöglichkeiten vieler Branchen und Tätigkeiten erweitert werden. Die kurzfristigen Jobverluste werden durch die erlebte Aufwertung der Arbeit ausgeglichen und ermöglichen einen Wiedereinstieg in die Berufswelt. „Jobs at every level would be enriched by an informating technology“ – alle bestehenden Tätigkeiten werden mit neuen Arbeitsinhalten angereichert, so können zum Beispiel Datenbrillen und Tablets die Qualifizierung der Beschäftigen verbessern, indem sie ihnen helfen, besseren Überblick über verschiedene Tätigkeiten zu erlangen und damit anspruchsvolle Berufe soweit zugänglich machen, dass sie von ehemals geringqualifizierten Beschäftigten ausgeübt werden können. Außerdem werden durch die Digitalisierung neue Kompetenzen gefordert, die ihren Weg in die Gesellschaft über eine Reform der Bildungsinhalte an Schulen und innerbetrieblichen Weiterbildungsmöglichkeiten finden. So sind zum Beispiel IT-Kompetenzen, Medienkompetenzen und Prozessverantwortung oder auch Arbeitsvorbereitung, Produktionsplanung und Qualitätssicherung Disziplinen, die zum Teil erst in einer vollständig digitalisierten Arbeitswelt von Bedeutung sind oder durch sie besser zugänglich gemacht werden können. Außerdem erleichtert Digitalisierung Konzepte wie z.B. Lerning-on-the-job oder Job Rotation, da verschiedene Berufe auch ohne fachspezifische Kenntnisse durchgeführt werden können, solange die notwendigen System beherrscht werden. Dadurch können ebenfalls vormals getrennte Arbeitsprozesse und ihre Verwaltung wieder vereint werden. \cite[S.183]{Eisenmann.2018}\\

In diesem Szenario ist klar erkennbar, dass die Sinnerfüllung durch die Digitalisierung bereichert wird. Durch die Erweiterung von Handlungsspielräumen können Aufgaben, die zuvor möglicherweise eintönig schienen und wenig Raum für das Einbringen subjektiver Ideen boten, so erweitert werden, dass das Individuum in der Lage ist, der Arbeit einen subjektiven Sinn zuzuschreiben. Kompetenzen, die bisher noch nicht von Bedeutung waren, werden neu gewichtet und bieten für die, die sie haben oder sie erlernen können ein Potenzial, das ohne Digitalisierung nicht möglich wäre. Außerdem werden Tätigkeiten, die bisher noch für entsprechend hochqualifiziertes Personal vorbehalten waren, zugänglich für geringqualifizierte Arbeiter. Einzelne Teilschritte können zusammengefasst werden und das Gefühl der Resonanz des eigenen Handelns wird wieder klarer.
\section{Polarisierung}
Das Szenario der Polarisierung ist sozusagen die Mischung der beiden zuvor besprochenen Szenarien. Hierbei findet eine partielle Substitution der Berufe statt, einige Berufe werden also durch Automatisierung überflüssig, andere wiederum erfahren eine Aufwertung durch den Einsatz von modernen Technologien. Dadurch klafft die Lücke zwischen Beschäftigten mit hohem Qualifikationsniveau und Einfacharbeitern auseinander, was zu der namensgebenden Polarisierung führt. Bei dieser Polarisierung stehen den Beschäftigten mit einem vielseitigen Set an Kompetenzen und Erfahrung die sogenannten Einfacharbeiter gegenüber, deren Beschäftigungen durch einen hohen Standardisierungsgrad und verminderte Entscheidungs- und Handlungsspielräume ausgezeichnet sind. Dabei erodiert die mittlere Qualifikationsebene, da sich manche Berufe in hoch qualifizierte aufwerten und manche substituiert werden. Zusätzlich dazu wird in den Berufen, die nicht ersetzt werden aber dennoch niedrig qualifiziert sind durch die Digitalisierung eine schärfere Kontrolle ermöglicht, was den Handlungsspielraum der Einfacharbeiter weiter einschränkt. Zusammenfassend gesagt, stehen in diesem Szenario die Dequalifizierungsprozesse und Substitutionsrisiken sowie Arbeitsregulierungen und Verengung der Handlungs- und Entscheidungsspielräume den Faktoren guter digitaler Arbeit und ihrer sinnvollen Ausgestaltung gegenüber\cite[S.183-184]{Eisenmann.2018}.\\

Da dieses Szenario die Kombination des Dystopie und der Utopie des sinnvollen Erlebens des Berufs in der Digitalisierung darstellt, sind entsprechend auch positive und negative Einflussfaktoren auf die Sinnerfüllung zu erkennen. Die Substitution birgt die Gefahr des Arbeitsplatzverlustes und den zuvor bereits geklärten Folgen für das Sinngefühl des Individuums. Verengung der Handlungsspielräume und Arbeitsregulierungen verhindern das Einbringen der subjektiven Aspekte in die Arbeit, wodurch eine Sinnerfüllung nahezu verhindert wird. Dem gegenüber stehen die Aufwertungen, die durch die Digitalisierung in den Berufen und Tätigkeiten einkehren. Hier ist also Chance und Risiko zugleich vorhanden und die Wirkung auf das Individuum hängt im Endeffekt davon ab, wie schnell es in eine der Branchen wechseln kann, die eine Aufwertung erfahren, um eine Sinnkrise verhindern zu können.
