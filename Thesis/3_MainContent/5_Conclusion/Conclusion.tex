\chapter{Fazit}
In dieser Arbeit wurde eine Sinndefinition ermittelt und daraufhin untersucht, wie sich Sinnerfüllung im Beruf erfahren lässt und an welchen Stellen diese gefördert oder gestört werden kann. Danach wurde betrachtet, wie die Digitalisierung in drei möglichen Szenarien zu einer Substitution, Aufwertung oder Polarisierung führt und wie sich diese Veränderungen auf die Sinnerfüllung im Beruf eines Individuums auswirken können. Zusammenfassend lässt sich sagen, dass sich durch die der Digitalisierung folgenden Veränderungen im Arbeitsalltag das Erleben von Sinn in jedem Szenario drastisch ändert. Es ist zum einen wichtig, die Risiken der Substitution und Einschränkungen besonders für geringqualifizierte Tätigkeiten zu beobachten und gegebenenfalls einzugreifen, um Sinnkrisen vorzubeugen. Dennoch sollte man eine Digitalisierung nicht gänzlich abblocken, da sich hier eine Vielzahl an Chancen bietet, die es ermöglicht die Sinnerfüllung im Beruf zu bereichern und für jedes Individuum spürbar sinnvolle Tätigkeiten zu vermehren und zu fördern. 