\chapter{Was macht Arbeit sinnvoll?}
[Einleitung yadda yadda]
Drei aspekte individuum und arbeit, qualität der organisation, arbeit und gesellschaft

Erstens soll die Arbeit ein Entgelt erbringen bzw. Profit erwirtschaften, die Arbeit selbst ist hier also ein Mittel für einen Zweck, ein Tausch-Wert. Zweitens soll das Produkt der Arbeit (hierzu zählt natürlich auch eine Dienstleistung) einen sozialen Nutzen erfüllen, sei es für einen bestimmten Kunden, Klienten oder Patienten oder sei es für einen gesellschaftlichen Zweck, für das Gemeinwohl. Hier geht es um den Gebrauchswert der Arbeit. [S.192-193 Fehlzeiten Report]
Der Sinn der Arbeit, ihr sozialer Nutzen, ist für Beschäftigte jedoch keineswegs unwichtig. Sie arbeiten täglich, ihr Beruf ist ein wesentlicher Teil ihrer Identität, sie werden von anderen ein Stück weit mit ihrer Arbeit identifiziert[S.193 Fehlzeiten Report]
Diese ist unmittelbar ersichtlich, wenn Altenpfleger pflegebedürftige ältere Menschen betreuen, wenn Paartherapeuten Paaren in ihrer Beziehung helfen wollen, wenn Installateure eine Heizung reparieren – immer dann also, wenn konkrete Kunden als Nutzer der Arbeit wahrnehmbar sind. Etwas schwieriger ist es, wenn die Nutzer in einem distanzierteren Verhältnis zu den Produzenten stehen – die Käufer von Autos, die Konsumenten von industriell hergestellten Fleischprodukten oder die Endkonsumenten von Produkten, deren Herstellung durch die Produktion von Maschinen ermöglicht werden soll.[Fehlzeiten Report S.193]
„Gefühl, an der Erstellung gesellschaftlich nützlicher Produkte beteiligt zu sein“ [Fehlzeiten Report, s.193]
Sie erwarten, dass die Arbeit so organisiert ist, dass sie (1) ihre „eigentliche Arbeit“ (2) effizient (3) in einer fairen Arbeitseinteilung (4) unter Einsatz ihrer Kompetenzen ausführen können. Eine besondere Bedeutung kommt hierbei einerseits (5) der Praxis der Vorgesetzten und (6) dem Verhältnis zu den Kollegen in der Arbeitskooperation zu[Fehlzeiten Report]


Antonovsky (1977), der Begründer des Konzeptes der Salutogenese, war einer der ersten, der diesen Zusammenhang empirisch bestätigen konnte. Grundlage dafür ist das Konstrukt des Kohärenzgefühls (sense of coherence, SOC), eine globale Orientierung, die ausdrückt, in welchem Ausmaß man ein durchdringendes, andauerndes und dennoch dynamisches Gefühl des Vertrauens hat. Antonovsky unterscheidet drei Komponenten des Kohärenzgefühls: 1. Die Verstehbarkeit (comprehensibility) bezieht sich „auf das Ausmaß, in welchem man interne und externe Stimuli als kognitiv sinnhaft wahrnimmt.“ 2. Die Handhabbarkeit (manageability) definierte Antonovsky als „das Ausmaß, in dem man wahrnimmt, dass man geeignete Ressourcen zur Verfügung hat, um den Anforderungen zu begegnen, die von den Stimuli ausgehen, mit denen man konfrontiert wird“. 3. Die Sinnhaftigkeit oder Bedeutsamkeit (meaningfulness) bezeichnete Antonovsky als die eigentlich emotionale Komponente des Konstrukts. Die Sinnhaftigkeit oder Bedeutsamkeit bezieht sich auf das Ausmaß, in dem die erwähnten Anforderungen „Herausforderungen sind, die Anstrengung und Engagement lohnen“ [S. 201, Fehlzeiten Report]
 

4 Sinnhaftigkeit wird durch Faktoren wie Partizipation und Entscheidungsspielraum beeinflusst (Antonovsky). 4 Sinnerfüllung ist eine wichtige Voraussetzung für Arbeitsengagement (Kahn, May et al.). 4 Sinnerfüllung und Arbeitsengagement sind zwei Seiten einer Medaille: Sinnerfüllung ist die kognitive Seite und Arbeitsengagement die affektivmotivationale Seite (Höge und Schnell). 4 Jeder Mensch kann Sinnhaftigkeit in seinem Beruf erleben. Er muss und kann den Sinn selbst finden und wird dann feststellen, dass es sich lohnt, Einsatz und Energie zu investieren (Frankl). 4 Sinnhaftigkeit erleben Menschen, die durch ihre Tätigkeit Ziele erreichen, die ihnen oder auch anderen Menschen wichtig oder wertvoll . Abb. 17.1 Einflussfaktoren auf das Sinnerleben im Beruf. (Quelle: eigene Darstellung; Kriterien in Anlehnung an Höge und Schnell 2012, Frankl 1987, Meller und Ducki 2002, Isaksen 2000) Fehlzeiten-Report 2018 Sinnerleben im Beru f Bewertung der Tätigkeit: herausfordernd, aber nicht überfordernd; Übereinstimmung mit persönlichen Werten, Beitrag zu einem großen Ganzen, talentfördernd Persönlichkeitsaspekte: Erfahrung der Selbstwirksamkeit, etwas Wichtiges oder Besonderes bewirken können, Verantwortung übernehmen, individuelle Motive Arbeitsumfeld: Anerkennung, Handlungsspielraum, positives Arbeits- und Betriebsklima, Zugehörigkeitsgefühl 206 Kapitel 17 · Sinnstiftung als Erfolgsfaktor 17 sind. Sie erleben sich als selbstwirksam (Tausch, Jeworrek). 4 Nicht nur Akademiker oder generell höher Qualifizierte können ihre Arbeit als sinnhaft erleben, sondern Angehörige aller Berufe, sei es, dass sie das Gefühl haben, einen wichtigen Beitrag zu leisten, jemandem zu helfen, den sie schätzen, zu einem Teil des Ganzen wurden oder Verantwortung übernahmen (Amabile und Kramer, Meller und Ducki, Isaksen). 4 Ob Beschäftigte einer Organisation ihre Arbeit als sinnhaft erleben, kann durch das Verhalten der Führungskräfte, das Arbeits- und Betriebsklima und die Unternehmenskultur beeinflusst werden (Badura und Walter, Echterhoff).

„noch in die restriktivste, monotonste und belastendste Arbeit subjektive Anteile der Arbeitenden eingehen, ja dass ohne diese subjektiven Anteile die Arbeit nicht leistbar wäre.“ (SenghaasKnobloch 2008, S. 79) Anhand dieser Bewältigungsformen versuchen die arbeitenden Menschen „ihre persönlichen Bedürfnisse und Sinnansprüche mit den Gegebenheiten in Einklang zu bringen.“
[Fehlzeitenreport s.66]
Ein Paradebeispiel für derartige Arbeitsstrukturierungsmaßnahmen sind teilautonome Arbeitsgruppen. Sie setzen auf job rotation (Aufgabenwechsel), job enlargement (Aufgabenvergrößerung) und job enrichment (erweiterte Entscheidungs- und Kontrollspielräume durch eine qualitative Anreicherung der Arbeit). [Fehlzeitenreport s.67]
